\documentclass[12pt, letterpaper]{article}

\usepackage{amsmath, amssymb}
\usepackage{physics}
\usepackage[french]{babel}
\usepackage[margin=2.5cm]{geometry}
\usepackage[utf8]{inputenc}
\usepackage[T1]{fontenc}
\usepackage{hyperref}
\usepackage{caption}

\begin{document}

\title{Devoir 1 : Équations différentielles ordinaires}
\author{PHQ404}
\date{8 février 2024}
\maketitle

\section{Objectif}\label{sec:objectif}

\noindent L'objectif de ce devoir est d'implanter plusieurs méthodes de solution
d'équations différentielles en Python 
et de les appliquer au problème de Kepler,
soit le mouvement d'un point soumis à une force attractive 
en inverse du carré de la distance.


\section{Comment présenter et remettre votre TP}\label{sec:comment-presenter-et-remettre-votre-tp}

\noindent Vous devez cloner le répertoire github dans l'organisation du cours au lien suivant :\\
\href{https://classroom.github.com/a/4PhLRaKv}{https://classroom.github.com/a/4PhLRaKv}.
Dans ce répertoire se trouvera votre code python, vos tests unitaires ainsi que votre rapport
décrivant les méthodes utilisés et l'analyse de vos résultats.
La structure des fichiers ne doit pas être modifiée, mais vous pouvez ajouter des fichiers si vous le désirez.
Voici la structure de fichiers que votre répertoire devra garder :

\bigskip

Root
\begin{itemize}
    \item[]
        \begin{itemize}
            \item[$\rightarrow$] src
                \begin{itemize}
                    \item[$\hookrightarrow$] \texttt{fichier0.py}
                    \item[$\hookrightarrow$] \texttt{fichier1.py}
                    \item[$\hookrightarrow$] \dots
              \end{itemize}
        \end{itemize}
  \item[]
  \begin{itemize}
    \item[$\rightarrow$] tests
    \begin{itemize}
      \item[$\hookrightarrow$] \texttt{test\_fichier0.py}
      \item[$\hookrightarrow$] \texttt{test\_fichier1.py}
      \item[$\hookrightarrow$] \dots
    \end{itemize}
  \end{itemize}
  \item[$\hookrightarrow$] \texttt{.gitignore}
  \item[$\hookrightarrow$] \texttt{requirements.txt}
  \item[$\hookrightarrow$] \texttt{README.md}
  \item[$\hookrightarrow$] \texttt{rapport\_devoir\emph{i}-\emph{vos\_noms}.pdf}
\end{itemize}

\bigskip

\noindent Le fichier \texttt{requirements.txt} doit contenir les dépendances de votre projet.
Le fichier \\\texttt{README.md} doit contenir les instructions pour installer et utiliser votre projet ainsi
qu'une brève description du devoir et des méthodes utilisés dans le code.
Le fichier \\\texttt{rapport\_devoir\emph{i}-\emph{vos\_noms}.pdf} doit contenir votre rapport en format pdf.
L'utilisation de \LaTeX\ est fortement recommandée pour la rédaction de votre rapport.
Dans le dossier \texttt{src} se trouvera votre code python et dans le dossier \texttt{tests} se trouvera vos tests
unitaires.

\bigskip

\noindent La remise et la correction automatique du code se fera à chaque \texttt{push} sur le répertoire github.
Notez que seul le dernier \texttt{push} sera considéré pour la correction.


\section{Énoncé}\label{sec:enonce}

\subsection{Implémentation}\label{subsec:implementation}

\noindent Dans le fichier \texttt{src/solvers.py}, vous devez implementer les fonctions suivantes pour résoudre une
équation différentielle ordinaire tout en conservant le même nom que ceux affichés ici-bas :
\begin{enumerate}
  \item \texttt{euler} pour la méthode d'Euler,
  \item \texttt{pred\_corr} pour la méthode du prédicteur-correcteur,
  \item \texttt{rk2} pour la méthode de Runge-Kutta d'ordre 2,
  \item \texttt{rk4} pour la méthode de Runge-Kutta d'ordre 4,
  \item \texttt{scipy\_int} pour la bibliothèque scipy
  (nous vous laissons regarder la documentation pour trouver une telle fonction). 
\end{enumerate}

\bigskip

\noindent Les \href{https://python-forge.readthedocs.io/en/latest/signature.html}{signatures} des fonctions doivent
respecter la suivante :
\begin{verbatim}
def solver_func(
        func: Callable[
            [Union[numbers.Number, np.ndarray], numbers.Number, Any], float
        ],
        y0: Union[numbers.Number, np.ndarray],
        time_steps: Union[numbers.Number, np.ndarray],
        *args,
        **kwargs
) -> np.ndarray
\end{verbatim}

\bigskip

\noindent Dans le fichier \texttt{src/systems.py}, vous devez implementer les fonctions suivantes pour décrire un
système d'équations différentielles ordinaires :
\begin{enumerate}
  \item \texttt{kepler} pour le problème de Kepler.
\end{enumerate}

\bigskip

\noindent Les \href{https://python-forge.readthedocs.io/en/latest/signature.html}{signatures} des fonctions doivent
respecter la suivante :
\begin{verbatim}
def system_func(
        y: Union[numbers.Number, np.ndarray],
        t: numbers.Number,
) -> np.ndarray
\end{verbatim}

\noindent Veuillez lire la documentation de chacune des méthodes pour comprendre les arguments.

\bigskip

\noindent Note: Toutes les fonctions qui doivent être implémentées sont déjà définies dans les fichiers
et retournent des \texttt{NotImplementedError}.

\subsection{Vérification}\label{subsec:verification}

Il est important de vérifer vos implémentations.
En effet, vous devez vous assurer que vos méthodes fonctionnent correctement et pour ce faire, vous devez rouler et
implémenter des tests unitaires qui testent chacune de vos classes et fonctions.
De plus, vous devriez tester si les résultats obtenus sont logiques.
Une façon de le faire, est d'appliquer vos méthodes à un système que vous pouvez résoudre analytiquement.
Par exemple, vous pouvez vérifier que vos fonctions retournent un résultat raisonnable pour un oscillateur harmonique en
2 dimensions, un projectile ou tout autre système simple.
Il serait aussi intéressant de retrouver ce type de vérification dans votre rapport.
Il est fortement recommandé d'ajouter des tests unitaires dans le dossier \texttt{tests}, mais les tests déjà
implémentés ne doivent pas être modifiés.

\subsection{Application au problème de Kepler}\label{subsec:application-au-probleme-de-kepler}

Supposez une particule de masse unité dans un espace en 2 dimensions soumise à une force attractive
\begin{align}
  \vb{f}(\vb r) = -\frac{1}{|\vb r|^2}.
\end{align}

Utilisez les quatre méthodes que vous avez implémentées pour résoudre ce système.
Ensuite, pour chacune des méthodes, 
tracer un graphique avec la position $x$ sur l'axe horizontal
et la position $z$ sur l'axe vertical.
Vous devez utiliser les conditions initiales suivantes:
\begin{align}
  &\vb{r_0} = (1.0, 0.0),
  &\vb{\dot{r}_0} = (0.0, 0.5).
\end{align}
De plus, le domaine d'intégration doit couvrir les temps de 0 
à 10 avec environ 2000 pas de temps.


\section{Critères d'évaluation}\label{sec:criteres-d'evaluation}

\begin{description}
  \item[70 points] Pour le résultat de vos tests unitaires.
  \item[30 points] Pour la qualité du rapport.
\end{description}


\end{document}
